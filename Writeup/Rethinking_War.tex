% !TEX TS-program = pdflatexmk

\documentclass[12pt]{article}
\usepackage{amsmath,amssymb}
\usepackage{graphicx}
\usepackage[margin=1in]{geometry}
\usepackage{mathptmx}
\usepackage{url}
\usepackage{subcaption}
\usepackage{changepage}
\usepackage{amsfonts}
\usepackage{booktabs}
\usepackage{siunitx}


\usepackage{stackengine}
\usepackage{dsfont}
\usepackage{verbatim}
\usepackage[scr=boondox]{mathalfa}
\newcommand{\bR}{{\mathbb R}}
\newcommand{\E}{{\mathbb E}}

\usepackage{mathptmx}
\title{Rethinking WAR for Pitchers: Grid WAR}
\author{Ryan Brill, Justin Lipitz, Emma Segerman, Ezra Troy, Adi Wyner}
\date{January 14, 2021}

\begin{document}
\maketitle

\section{Motivation}



% CITATIONS
%https://www.espn.com/mlb/player/gamelog/_/id/28976/year/2014/category/pitching
% https://www.fangraphs.com/leaders.aspx?pos=all&stats=sta&lg=all&qual=y&type=8&season=2014&month=0&season1=2014&ind=0&team=0&rost=0&age=0&filter=&players=0&startdate=2014-01-01&enddate=2014-12-31
% https://library.fangraphs.com/war/calculating-war-pitchers/
% https://www.baseball-reference.com/about/war_explained_pitch.shtml



Traditional methods for computing WAR (wins above replacement) for pitchers, notably by Baseball Reference and Fangraphs, calculate WAR as a function of a pitcher's average performance. Specifically, in computing WAR, Baseball Reference averages a pitcher's performance over the course of a season via its $xRA$ statistic [CITE]. $xRA$, or ``expected runs allowed'', denotes a pitcher's average number of runs allowed per out. Additionally, Fangraphs averages a pitcher's performance over the course of a season via its $ifFIP$ statistic [CITE]. $ifFIP$ , or ``fielding independent pitching (with infield flies), is defined by
$$ifFIP := \frac{13\cdot HR + 3\cdot(BB+HBP) - 2\cdot(K+IFFB)}{IP} + ifFIP constant,$$
which involves averaging some of a pitcher's statistics over his innings pitched. In this paper, we argue that using a pitcher's average performance to calculate his WAR is a bad way to measure his value on the mound, and so we derive an alternative approach to computing WAR.  


Consider Max Scherzer's six game stretch from June 12, 2014 through the 2014 all star game [CITE]. His performance over those six games is shown in table (\ref{Tab:p1}).  \\

\begin{table}[ht]
\centering
\begin{tabular}{|r|cccccc|} \hline
\text{game} & 1 & 2 & 3 & 4 & 5 & 6\\ \hline
\text{earned runs} & 0 & 10 & 1 & 2 & 1 & 1 \\
\text{innings pitched} & 9 & 4 & 6 & 7 & 8 & 7 \\ \hline
\end{tabular}
\caption{Max Scherzer's performance over six games prior to the 2014 all star break.}
\label{Tab:p1}
\end{table}

In Scherzer's six game stretch, he averages 15 runs over 45 innings, or $1/3$ runs per inning. So, on average, Scherzer pitches three runs per complete game, which isn't great. If we look at each of Scherzer's individual games separately, however, we see that he has five dominant performances and one blowup, which is essentially equivalent to five wins and one loss. Pitching three runs per complete game is certainly not as stellar as five nearly-guaranteed wins and one loss. On this view, a WAR statistic based upon averaging a pitcher's performance would significantly devalue Scherzer's value during this six game stretch. Because ``you can only lose a game once'', it makes more sense to give Scherzer zero credit for his one blowup game than to distribute his one poor performance over all his other games via averaging. Hence we should not compute WAR as a function of a pitcher's average performance, and should compute season-long WAR as the summation of the WAR of his individual games. 

Another way of thinking about why it is problematic to use earned run average to compute WAR is that WAR should be a nonlinear function. In other words, if $R_j$ denotes a pitcher's number of runs allowed in inning $j \in \{1,...,n\}$ of the season, then we expect
\begin{equation}
WAR\bigg( \sum_{j=1}^{n} R_j, n \bigg) \neq  \sum_{j=1}^{n} WAR(R_j, 1).
\label{eqn:war1}
\end{equation}
Equation (\ref{eqn:war1}) tells us that, if you think of WAR as a function $R$ and $I$, which refers to a pitcher allowing $R$ runs over $I$ innings, then the WAR of a pitcher's total runs allowed and total innings over the season is not the same as the summation of a pitcher's WAR over his individual games.

As discussed in the Scherzer 2014 example, a good WAR statistic should be a function of a player's individual game performances, as depicted in the right hand side of equation (\ref{eqn:war1}). Traditional WAR statistics, as computed by Baseball Reference and Fangraphs, however, implicitly treat WAR as a linear function by using a WAR statistic of the form of the left side of equation (\ref{eqn:war1}). 




\section{Grid War: Model Specification}

\subsection{Overview}

\subsection{Win Probability Added}

In order to define the win probability added by pitcher $p$ during game $i$, we need to first define two helper functions. First, we define the function $f = f(I,R)$ which computes a team's probability of winning a baseball game after giving up $R$ runs through $I$ complete innings (averaged over all other confounders such as the performance of his team's batters). Second, we define the function $g = g(R,S,O)$ which computes the probability that, starting midway through an inning with $O \in \{0,1,2\}$ outs and men on base given by $S \in \{000,100,010,001,110,101,011,111\}$, a team scores exactly $R$ runs through the end of the inning. Because $f$ is a function of two discrete variables $I \in \{1,2,...,9\}$ and $R \in \{0,1,...27\}$ (where 27 serves as the maximum number of runs allowed to be pitched in a game), $f$ is fully specified by a 2D grid. Similarly, if we combine $S$ and $O$ into a 24-state tuple, then we may also view $g$ as a function of two discrete variables $R$ and $(S,O)$, and so $g$ is also fully specified by a 2D grid. As we end up defining a pitcher's WAR in terms of these helper functions $f$ and $g$, which are specified by 2D grids, we decide to name our metric \textit{Grid WAR}. 

Now, using these helper functions $f$ and $g$, we wish to compute a pitcher's win probability added during a game. We begin by defining a pitcher's win probability added during a stretch of complete innings pitched. Suppose a pitcher begins pitching at the start of inning $I_0$, and that his team has already given up $R_0$ runs. Then suppose he pitches a number of complete innings, and exits the game at the end of inning $I_1$ having thrown $R_1$ runs. Then we define his win probability added during this time as
$$f(I_1, R_0+R_1) - f(I_0 - 1, R_0).$$
We use the convention that $f(I=0,\cdot) = 1/2$.

We now define a pitcher's win probability added during a partial inning.  Suppose a pitcher begins pitching at the start of inning $I$, and that his team has already given up $R_0$ runs.  Then suppose he gives up $R_1$ runs, and exits the game midway through inning $I$ with base-state $S$ and number of outs $O$. Then we define his win probability added during the beginning-of-this-inning ($boi$) as
$$boi(I,R_0,R_1,S,O) := \sum_{r \geq 0} g(r,S,O) f(I,R_0+R_1+r)   - f(I-1,R_0).$$

Alternatively, suppose a pitcher begins pitching midway through inning $I$, starting with base-state $S$ and number of outs $O$. Then suppose he exits the game at the end of inning $I$ having given up $R_1$ runs. Suppose also that his team gives up $R_0$ runs through inning $I$, and $R_0'$ runs during inning $I$ but before he begins pitching. Then we define his win probability added during the end-of-this-inning ($eoi$) as
$$eoi(I,R_0,R_0',R_1,S,O) := f(I,R_0+R_0'+R_1) - f(I-1,R_0) - boi(I,R_0,R_0',S,O) .$$

Lastly, suppose a pitcher begins pitching midway through inning $I$, starting with base-state $S_0$ and number of outs $O_0$. Then suppose he gives up $R_1$ runs, and exits the game with base-state $S_1$ and number of outs $O_1$. Suppose also that his team gives up $R_0$ runs through inning $I$, and $R_0'$ runs during inning $I$ but before he begins pitching. Then we define his win probability added during the middle-of-this-inning ($moi$) as
\begin{align*}
moi(I,R_0,R_0',R_1,S_0,O_0,S_1,O_1) := & f(I,R_0+R_0'+R_1) - f(I-1,R_0) \\
&- eoi(I,R_0,R_0',R_1,S_1,O_1) - boi(I,R_0,R_0',S_0,O_0).
\end{align*}

Now, we have defined the win probability added by a pitcher during any inning, whether it is a complete or partial inning. Then, we define the win probability added by a pitcher over the course of the season as the sum of the win probability added over all of his innings pitched. Now, we need only estimate the helper functions $f$ and $g$ from the data in order to compute each pitcher's win probability added over a season. 

\subsection{Wins Above Replacement \& the Cascading Replacement Paradigm}

We have formulated a method to compute a pitcher's win probability added over the course of a season, and we shall use this to define Grid WAR (GWAR). A WAR metric is intended to compare a pitcher to a \textit{replacement}-level pitcher, which is someone who would substitute for a pitcher if he got injured. We wish to define a pitcher's GWAR by the difference in his win probability added over the course of a season and that of a replacement-level player. Hence we need to quantify the performance of a replacement-level pitcher.

In general, the replacement for a starter is a low-tier starter, the replacement for a closer is a late-inning-middle-reliever, the replacement for a late-inning-middle-reliever is a mid-game-middle-reliever, and the replacement for a mid-game-middle-reliever is a low-tier mid-game-middle-reliever. Naturally, we call this system of replacement the \textit{cascading replacement paradigm}. 

%To this end, we create four archetypical replacement level pitchers: the low-tier starter, the low-tier middle-reliever, the average middle-reliever, and the average mid-late-inning-reliever. For each of these archetypes $a \in \{1,2,3,4\}$, we create a grid function $h_a = h_a(R)$ which computes archetype $a$'s probability of throwing $R$ runs during a complete inning, and a grid function $k_a = k_a(R,S,O)$ which computes archetype $a$'s probability of throwing $R$ runs from the middle of an inning with base-state $S$ and number of outs $O$ through the end of the inning. Note that $h_a(R) - k_a(R,S,O)$ gives the probability that archetype $a$ throws $R$ runs from the start of an inning to the middle of an inning if he exits with base-state $S$ and number of outs $O$. Further, note that $h_a(R) - k_a(R,S_1,O_1) - k_a(R,S_1,O_1)$ gives the probability that archetype $a$ throws $R$ runs during the middle of an inning, starting with base-state $S_0$ and number of outs $O_0$, and ending with base-state $S_1$ and number of outs $O_1$.



%To this end, we create four archetypical replacement level pitchers: the low-tier starter, the low-tier middle-reliever, the average middle-reliever, and the average mid-late-inning-reliever. Let $r_a$ denote the expected number of runs by archetype $a\in \{1,2,3,4\}$ in a complete inning. Let $r_a(S,O)$ be the expected number of runs by archetype $a$ starting midway through an inning with base-state $S$ and number of outs $O$ until the end of the inning. Note that $r_a - r_a(S,O)$ is the expected number of runs by archetype $a$ from the beginning of an inning until midway through an inning if he exits with base-state $S$ and number of outs $O$. Further, note that $r_a(S_1,O_1) - r_a(S_0,O_0)$ is the expected number of runs by archetype $a$ starting midway through an inning with base-state $S_0$ and number of outs $O_0$ until he is pulled midway through this inning with base-state $S_1$ and number of outs $O_1$. 




\section{Computing the Grid Functions}

\subsection{Estimating $f$}


\subsection{Estimating $g$}


\subsection{Replacement-Level Computations}



%\subsection{Estimating $h_a$}
%
%
%\subsection{Estimating $k_a$}



\section{Examples}



\section{Conclusion}

Future Work: potentially modify win probability added to include other confounder's like the pitcher's team's runs, opposing fielder quality, etc

much better way to compute WAR!


\end{document}